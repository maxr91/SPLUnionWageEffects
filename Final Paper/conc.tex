\section{Conclusions}\label{Sec:Conc}
In this paper, we analyze the effect of collective bargaining coverage on wages and its implications for wage dispersion in Germany. Our analysis distinguishes between individual coverage on the sectoral level and firm-level and the within-firm coverage ratios for both bargaining regimes. The econometric investigation is conducted using a linked employer-employee dataset, the German Structure of Earnings Survey 2010, provided by the Research Data Centers of the Federal Statistical Office and the statistical offices of the L\"ander. Throughout my analysis we control for individual and firm characteristics in order to reduce the endogeneity problem of collective coverage, i.e. the selection bias based on observable characteristics, but of course, we cannot rule out selection on unobservable characteristics. For example, there might be a selection bias with regard to the productivity distribution in the covered and uncovered sector since the more productive workers should have a preference for the uncovered sector according to the \emph{worker-choice model} \citep{Lee:78}. An alternative selection bias might occur when employers hire more highly productive employees in response to the presence of a union, adapting to the high wages of less-qualified workers. Combining the two examples yields that the unionized sector is mainly composed of employees with an average productivity. Highly productive workers refuse to be unionized and low productive workers will not be hired. Additionally, the assumption of an exogenous coverage ratio could be violated if workers especially try to organize in industries with potentially high gains from unionization. 

Using OLS regression as well as conditional and unconditional quantile regressions, we come to the following conclusions. The share of covered employees within a firm has a positive effect on wages and increases along the unconditional wage distribution. Thus, firms that apply a collectively negotiated contract pay higher wages than uncovered firms, particularly benefitting high-wage earners. This finding suggests that a higher share of covered employees within a firm contributes to wage dispersion. \cite{Fitzenberger&Kohn&Lembcke:13} confirm the positive effect on wages, but find a fairly constant impact across conditional quantiles of the wage distribution. Our conditional quantile regression results are in line with the estimates obtained by \cite{Fitzenberger&Kohn&Lembcke:13}. Holding the coverage share in the firm constant, individual coverage under sectoral and firm-level contracts show a positive effect for lower quantiles, which turns negative towards the top of the unconditional wage distribution. Thus, individual coverage by any of the two coverage regimes reduces wage dispersion.

Since wage inequality in many industrialized countries, including Germany, has risen over the last couple of decades and union density as well as collective bargaining coverage declined over the same period, future research should further investigate the causal relationship between the two developments, based on \cite{Antonczyk&&&:11} and \cite{Dustmann&Ludsteck&Schönberg:09}.

