\section{Introduction}
The existence of labor market unions is certainly one of the major departures from the market wage-setting mechanism. By utilizing their bargaining power and restricting labor supply, unions may achieve above market wages. Moreover, the union bargaining results affect wage dispersion, because wages are attached to jobs rather than to employees \citep{Bryson:14}. However, the size and quality of union wage effects depend on a variety of factors such as the bargaining power of unions, the institutional level of collective bargaining and the factors included in the bargaining process, all of which make it a heavily contested field of research.

In this paper we will estimate the effect of collective bargaining coverage on mean wages for covered employees in Germany using a linked employer-employee dataset, the German Structure of Earnings Survey (GSES) 2010. Then, conditional and unconditional quantile regressions are employed in order to estimate the effect of collective bargaining on wage dispersion. Throughout the theoretical considerations and our econometric investigation, we distinguish between individual coverage on the sectoral and firm-level as well as the within-firm collective bargaining coverage ratio for both collective bargaining regimes.

It proves important to distinguish between union density and collective bargaining coverage. Union density refers to the share of employees that are unionized, whereas collective bargaining coverage comprises all employees that are covered by a collective agreement. As can be seen in figure \ref{Fig:UDandCC2}, union density and collective bargaining coverage can greatly differ from one another within one country. In addition, there is a declining trend in trade union density as well as collective bargaining coverage. Whereas union density rose in Spain, it decreased in all other OECD countries, that are included in figure \ref{Fig:UDandCC2}, from 1990 to the latest year recorded.\footnote{Please see appendix B Figure \ref{Fig:UDandCC2} for a breakdown of the country-specific latest years of record.} The OECD average of collective bargaining coverage declined as well, with disproportionately high losses in countries with lower coverage ratios in 1990.

The institutional landscape of unions varies across countries, affecting the gap between union density and collective bargaining coverage. Employees may be covered by sectoral wage bargaining agreements, which are negotiated between an employer's association and an employee union. A second possible form of coverage is firm-level coverage, which results from bargaining between a single firm and an employee union. In Germany, these two forms of bargaining are mutually exclusive. In the United States, for example,  bargaining predominantly takes place at the firm-level and union density and the share of covered employees are relatively similar (Fig. \ref{Fig:UDandCC2}). Conversely, in Germany and other European countries, bargaining is primarily conducted at the sectoral level and the coverage rate generally exceeds union density (Fig. \ref{Fig:UDandCC2}). International differences in the institutional setup and country-specific legal settings explain the disparities between collective bargaining coverage and union density. For example, in France and Spain, employers are not allowed to discriminate against non-union members, compared to unionized employees. Therefore, collective contracts have to be extended to non-union members and as a result collective coverage exceeds union density. This depicts a free-rider problem with regard to union membership and explains the low rates of union density relative to collective coverage in some countries. In countries where discrimination between unionized and non-unionized employees is legal, such as the United States (US) and the United Kingdom (UK), union density is considerably higher relative to collective coverage. In Germany, employers often voluntarily extend collective contracts to non-unionized workers. Following from the German Collective Bargaining Act,\footnote{ Tarifvertragsgesetz} collective contracts have to be applied to a specific job match, only if the employer is part of an association and the employee is a member of a union. Hence, the extension is not forced by law, but a response to the bargaining power of employees \citep{Fitzenberger&Kohn&Lembcke:13}. However, there is some room for deviation since employers are always free to pay higher wages than collectively negotiated (favorability principle). Furthermore, individual wages of covered employees may differ as each employee has the right not to associate, according to the German constitution.\footnote{ negative Koalitionsfreiheit}

The results of our econometric investigation show a positive union wage effect of individual coverage for low-wage earners under both coverage regimes. The effect declines along the wage distribution and eventually turns negative for high-wage earners, which indicates a compression of the wage distribution. Furthermore, an increase in the share of covered employees increases wages in the firms that apply either sectoral or firm-level contracts.

The aim of this paper is to present an empirical paper with the use of the statistical programming language R. The paper is organized as follows. Section \ref{Sec:Data} provides information of the raw data. Section \ref{Sec:Data_Prep} is a process to improve the quality of the data information and to put raw data into the form so that we can work scientifically with these data. In this section we will carry out some standardization of the variables, as well as transformations and calculations of new variables. Section \ref{Sec:Des_Stat} shows how to display tables and graphics using own functions. In order to analyze the above-described economic relations, we are using different regression analysis methods in section \ref{Sec:regressionR}. Then in section \ref{Sec:Conc} the statistical programming results are summarized.
