\section{Code}

Please note that the quantlets are all related to their order and are specifically adapted to the data set \texttt{dataset2010.dta}. The quantlets must be executed in their order to obtain the same results as presented in this paper.


\subsection{Code for Data Preparation}

\subsubsection{Code for Quantlet 1}
\lstset{firstnumber = 1}
\begin{lstlisting}
#clear the workspace
rm(list=ls())

#install package / load library for importing stata 13 files
install.packages("readstata13")
library(readstata13)

#importing the data into dat
dat = read.dta13("dataset2010.dta", convert.factors = TRUE, 
	generate.factors = FALSE, encoding = "UTF-8",  fromEncoding = NULL,  
	convert.underscore = FALSE, missing.type = FALSE, convert.dates = TRUE, 
	replace.strl = TRUE, add.rownames = FALSE, nonint.factors = FALSE,
	select.rows = NULL)

#function for calculation respondents per company
respondFunc = function(dat){
  if (missing(dat)) 		stop("No data passed to the function")
  if (is.numeric(dat)!= TRUE)	stop("Numeric data needed")
  respond = numeric(length(dat))
  i = 1                                   #setting counting variables to 1
  j = 1
  temp = table(dat)                       #how many different values are in dat
  for (i in 1:length(dat)){               #for every observation
    j = dat[i]                            #store value of dat in j
    respond[i] = temp[j]                  #value of temp[i] store in respond[i]
    i = i+1              
  }
  return(respond)
}

respond = respondFunc(dat$ef1)			#call function respondFunc
dat["respond"] = respond			#add to data frame

#dummyfunction to create dummy variables,
#compare data vector with selected level of data vector, 
#if true, then function writes 1
dummyFunc = function(dat , x){
  if (missing(dat))			stop("No data passed to the function")
  if(is.null(levels(dat)))		stop("No levels found")
  d = as.numeric(dat == levels(dat)[x])      
  return(d)
}

#create eastdummy  0=west
east = dummyFunc(dat$ef4be , 5)  		#call function dummyFunc
dat["east"] = east           	    		#add to data frame

#create less classes for education
tempEdu1  = dummyFunc(dat$ef16u2 , 1 )
tempEdu1a = dummyFunc(dat$ef16u2 , 2 )
tempEdu2  = dummyFunc(dat$ef16u2 , 3 )
tempEdu2a = dummyFunc(dat$ef16u2 , 4 )
tempEdu3  = dummyFunc(dat$ef16u2 , 5 )
tempEdu3a = dummyFunc(dat$ef16u2 , 6 )
tempNa    = dummyFunc(dat$ef16u2 , 7 )
tempNa[tempNa == 1] = NA               		 #add NA's from dataset

#reduce dummy levels from 6 to 3
educ1 = tempEdu1 + tempEdu1a + tempNa   
educ2 = tempEdu2 + tempEdu2a + tempNa
educ3 = tempEdu3 + tempEdu3a + tempNa

dat["educ1"] = educ1                  	 	#add to data frame
dat["educ2"] = educ2
dat["educ3"] = educ3

#create dummy for permanent workers
permanent        = dummyFunc(dat$ef17 , 1 )
dat["permanent"] = permanent        		 #add to data frame

#define whether someone worked in shifts/at night/...
shift        = as.numeric(dat$ef23 >= 1)
dat["shift"] = shift                   		 #add to data frame

#create dummy for fulltime workers and reduce levels 
tempFull1 = as.numeric(dat$ef16u1 != "Teilzeitbeschäftigt - Beamter")
tempFull2 = as.numeric(dat$ef16u1 != "Teilzeitbeschäftigt - weniger als 18 Std.")
tempFull3 = as.numeric(dat$ef16u1 != "Teilzeitbeschäftigt - 18 Std. und mehr")
fulltime  = tempFull1+tempFull2+tempFull3 - 2
dat["fulltime"] = fulltime        		   #add to data frame

#create minimumwage dummy 0=nein 
minimumWage   = as.numeric(dat$ef31be != "nein")
minimumWageNa = dummyFunc(dat$ef31be , 3 )
minimumWageNa[minimumWageNa == 1] = NA 	 		#add NA's from dataset
minimumWage        = minimumWage+minimumWageNa
dat["minimumWage"] = minimumWage + minimumWageNa       #add to data frame
\end{lstlisting}


\subsubsection{Code for Quantlet 2}
\lstset{firstnumber = 89}
\begin{lstlisting}
#create function to calculate how many employees have a union contract or not
#a= company dummy, b= data information vector with dummies, c= choose level
contractFunc = function(a,b,c){            
  if (missing(b)) 		stop("No data passed to the function")
  if (is.numeric(a)!= TRUE & is.numeric(b)!= TRUE  )
    stop("numeric data needed")
  if (missing(c))		stop("No level selected")
  temp = table(a)
  #calculate cummulative sums for later addressing the vector
  cumtemp = cumsum(temp)                 
  cumtemp = append(cumtemp,0,after =0)     #need 0 as start value 
  g = nrow(temp) +1
  i = 2                                    #setting counting variables to 1
  j = 1
  k = 1
  q = numeric(length(b))
  for (i in 2:g){
    k = cumtemp[i-1]+1                   #store starting value company x in dat in k
    j = cumtemp[i]                       #store end value company x in dat in j
    #store results in p (how many people have kein Tarifvertrag and so on....)
    p = table(b[k:j])                      
    q[k:j]  = p[c]                   
    i = i+1
  }
  return(q)
}

noTariff = contractFunc(dat$ef1, dat$ef8, 1) 	#call function contractFunc
SCTariff = contractFunc(dat$ef1, dat$ef8, 2)
FCTariff = contractFunc(dat$ef1, dat$ef8, 3)
dat["noTariff"] = noTariff               	 #add to data frame
dat["SCTariff"] = SCTariff
dat["FCTariff"] = FCTariff

#create shares
shareFC = FCTariff/respond
shareSC = SCTariff/respond
dat["shareFC"] = shareFC           		 #add to data frame
dat["shareSC"] = shareSC

#create dummy variables for no labor, sectoral and firm-level contract
noTariffDummy = dummyFunc(dat$ef8 , 1)
SCTariffDummy = dummyFunc(dat$ef8 , 2)
FCTariffDummy = dummyFunc(dat$ef8 , 3)
dat["noTariffDummy"] = noTariffDummy          	#add to data frame
dat["SCTariffDummy"] = SCTariffDummy
dat["FCTariffDummy"] = FCTariffDummy

#create interaction terms
shareFCFC = shareFC*FCTariffDummy
shareSCSC = shareSC*SCTariffDummy
dat["shareFCFC"] = shareFCFC          		#add to data frame
dat["shareSCSC"] = shareSCSC

#create variables age squared and experienece squared 
agesq = dat$ef40*dat$ef40
expsq = dat$ef41*dat$ef41
dat["agesq"] = agesq         			#add to data frame
dat["expsq"] = expsq

#define wage and lnwage
wage   = ifelse(dat$ef18+dat$ef20 == 0, NA, 
	(dat$ef21+dat$ef22)/(dat$ef18+dat$ef20))
lnWage = log(wage)
dat["wage"]   = wage              		#add to data frame
dat["lnWage"] = lnWage
\end{lstlisting}

\subsection{Code for Descriptive Analysis}
\subsubsection{Code for Quantlet 3}
\lstset{firstnumber = 156}
\begin{lstlisting}
#install and load plotrix-package / neccessary to use pyramid.plot
install.packages("plotrix")
library("plotrix")

#function to calculate relative frequencies in % table for variable k with l different characteristics
frequency = function(k, l){            
  if (missing(k))
    stop("No data passed to the function. Variable k has to be defined.")
  if (missing(l))
    stop("No data passed to the function. Variable l has to be defined.")
  100*sweep(table(k,l), 2, colSums(table(k,l)), "/")
}  

#function to build population pyramid and store it as pdf
buildpopulation = function(k, l, popname){
  if (missing(popname))
    stop('No data passed to the function. Variable popname has to be defined.
         Please define a plot name such as "populationpyramid.pdf". 
         Use quotation marks, at the beginning and the end of the plot name.')
  pop = frequency(k, l)
  pdf(popname)
  pyramid.plot(pop[,1], pop[,2], labels = rownames(pop), gap = 2,
               lxcol = "blue", rxcol = "red")
  dev.off() 
}

#subsample with employees covered by the union contract (with FC or SC)
datFCSC = dat[ which(SCTariffDummy == 1 | FCTariffDummy == 1),]

#subsample with employees not covered by the union contract (no FC & no SC)
datNoFCSC = dat[ which(SCTariffDummy == 0 & FCTariffDummy == 0),]

#population pyramid on the whole dataset
buildpopulation(dat$ef41, dat$ef10, "population_all.pdf")

#population pyramid of employees covered by the union contract
buildpopulation(datFCSC$ef41, datFCSC$ef10, "populationFCSC.pdf")

#population pyramid of employees not covered by the union contract
buildpopulation(datNoFCSC$ef41, datNoFCSC$ef10, "population-noFCSC.pdf")

#calculate arithmetic mean
mean(dat$ef41, na.rm = TRUE)          #all population
mean(datFCSC$ef41, na.rm = TRUE)      #subsample - union covered workers
mean(datNoFCSC$ef41, na.rm = TRUE)    #subsample - workers without a union contract

#calculate median
median(dat$ef41, na.rm = TRUE)          #all population
median(datFCSC$ef41, na.rm = TRUE)      #subsample - union covered workers
median(datNoFCSC$ef41, na.rm = TRUE)    #subsample - workers without a union contract

#function to simultaneously generate and save boxplot in a pdf-file
buildboxplot =  function (v, w , boxname, z){
  if (missing(v))
    stop("No data passed to the function. Variable v has to be defined.")
  if (missing(w))
    stop("No data passed to the function. Variable w has to be defined.")
  if (missing(z))
    stop("No data passed to the function. Variable z has to be defined.
         z is a vector which should contain labels for the characteristics of 
         the variable w. The number of the characteristics of w must equal the 
         number of elements in z.")
  if (missing(boxname))
    stop('No data passed to the function.boxname has to be defined such as 
    	"graph.pdf".') 
  if (is.numeric(v)!= TRUE)
    stop("Numeric data needed. k has to be a numeric variable.")
  pdf(boxname, width = 11, height = 7)
  boxplot(v~w, range=2.5, width=NULL, notch=FALSE,varwidth=FALSE, names = z,
          boxwex=0.8, outline=FALSE, staplewex=0.5, horizontal=FALSE, 
          border="black", col="#94d639", add=FALSE, at=NULL)          
  abline(h = median(v, na.rm = TRUE), col="red", lwd = 1.5)
  dev.off()
}

#define a vector with label names for gender and education
genderLAB = c("male", "female")
educLAB = c("Educ A", "Educ B", "Educ C", "Educ D", "Educ E", "Educ F", 
		"Educ G")

#boxplot ln(wage)~gender of all employees
buildboxplot(dat$lnWage, dat$ef10, "boxplot_lnwage_gen.pdf", genderLAB)

#boxplot ln(wage)~education (educ) of all employees 
buildboxplot(dat$lnWage, dat$ef16u2, "boxplot_lnwage_educ.pdf", educLAB)

#boxplot ln(wage)~educ of employees which are covered by an union contract 
buildboxplot(datFCSC$lnWage, datFCSC$ef16u2, "boxploteducFCSC.pdf", educLAB)

#boxplot ln(wage)~educ of employees which are not covered by an union contract
buildboxplot(datNoFCSC$lnWage, datNoFCSC$ef16u2, 
	"boxploteduc-NoFCSC.pdf", educLAB)

#calculate median of the variable ln(wage)
median(dat$lnWage, na.rm = TRUE)
median(datFCSC$lnWage, na.rm = TRUE)
median(datNoFCSC$lnWage, na.rm = TRUE)
\end{lstlisting}

\subsubsection{Code for Quantlet 4}
\lstset{firstnumber = 254}
\begin{lstlisting}
#function to calculate quantiles
quant = function(y, x, q){
  if (missing(x))
    stop("No data passed to the function. Variable x has to be defined.")
  if (missing(y))
    stop("No data passed to the function. Variable y has to be defined.")
  if (missing(q))
    stop("No data passed to the function. Variable q has to be defined.")
  if (is.numeric(y)!= TRUE)
    stop("Numeric data needed. y has to be a numeric.")
  if (is.numeric(q)!= TRUE)
    stop("Numeric data needed. Quantile q was wrong specified, 
    	q can be either a value or a numeric value.")
  aggregate(y, list(x), na.rm=TRUE, quantile, q)
}

#define a vector with quantiles
q = c(0.10, 0.25, 0.50, 0.75, 0.90) 

#define a vector with used colors
color = c("orange", "red", "green", "blue", "black")

#Function to construct scatterplot with quantile lines
buildquantileplot = function(x, y, xla, yla, plotname){
  if (missing(xla))
    stop("No data passed to the function. Variable xla has to be defined.
         This is a label for the x-axis.")
  if (missing(yla))
    stop("No data passed to the function. Variable yla has to be defined.
         This is a label for the y-axis.")
  if (missing(plotname))
    stop('No data passed to the function.
    	plotname has to be defined such as "graph.pdf".') 
  pdf(plotname)
  #plot points
  plot(x, y, ylim=c(2,8), pch = 1, col="dark green", 
       xlab = xla, ylab = yla)
  #plot quantilelines
  for (l in 1:length(q)){
    lines(quant(y, x, q[l]), col = color[l], lwd =2)
  }
  dev.off()
}

#Scatterplot with quantile-lines ln(wage)~age
buildquantileplot(dat$ef41, dat$lnWage, "Age", "Ln(wage)", 
	"scatterplot_lnwage_age.pdf")

#Scatterplot with quantile-lines ln(wage)~experience
buildquantileplot(dat$ef40, dat$lnWage, "Experience", "Ln(wage)", 
	"scatterplot_lnwage_experience.pdf")

#Scatterplot with quantile-lines ln(wage)~experience
buildquantileplot(datFCSC$ef40, datFCSC$lnWage, "Experience", "Ln(wage)",
	"scatterplotFCSC_lnwage_experience.pdf")

#Scatterplot with quantile-lines ln(wage)~experience
buildquantileplot(datNoFCSC$ef40, datNoFCSC$lnWage, "Experience", "Ln(wage)",
	"scatterplotNoFCSC_lnwage_experience.pdf")

\end{lstlisting}

\subsubsection{Code for Quantlet 5}
\lstset{firstnumber = 314}
\begin{lstlisting}
#install and load data.table-package 
install.packages("data.table")    
library(data.table)   

#convert data frame into data table       
dat = data.table(dat)             		

#create group with 3 factors
dat[SCTariffDummy == 1, Group := factor(1)]   
dat[FCTariffDummy == 1, Group := factor(2)]
dat[noTariffDummy == 1, Group := factor(3)]

#name each factor
dat[, Group := factor(Group, labels = c("SC", "FC", "IC"))]	

#how many observations in total are in the data table (without nas)
sum = dat[!is.na(Group), .N]                       

#calculate mean and standard deviation of lnwage for each group and each gender
lnWageSummary = dat[!is.na(Group), .(LogHourlyWageMean = mean(lnWage, na.rm = T),
	LogHourlyWageSD = sd(lnWage, na.rm = T)), by = .(ef10, Group)]  

#order variables according to gender and group
lnWageSummary = lnWageSummary[order(ef10, Group)]               

#calculate total mean (no discrimination between gender)                     
lnWageSummaryOverall = dat[!is.na(Group), .(LogHourlyWageMean = mean(lnWage, na.rm = T),
	LogHourlyWageSD = sd(lnWage, na.rm = T)), by = .(Group)]

#order variables according to group
lnWageSummaryOverall = lnWageSummaryOverall[order(Group)]

#calculate frequencys for every group and by gender
mtable = table(dat$Group, dat$ef10)                                            

#calculate employee share (no discrimination between gender)
TotalEmpolyeeShare = dat[!is.na(Group), .(Share = .N/sum), by = .(Group)] 

#order values    
TotalEmpolyeeShare = TotalEmpolyeeShare[order(Group)]                          

#create full table with before calculcated values 
lnWageSummaryTotal = data.frame(Regime = c("SC", "FC", "IC"),
		#calculate proportion for employee share (male)                                       
	MaleEmpolyeeShare = prop.table(mtable, 2)[, 1], 
		#put male wage mean value 
	MaleLogHourlyWageMean = lnWageSummary[ef10 == "männlich", LogHourlyWageMean], 
		#put male standard deviation   
	MaleLogHourlyWageSD = lnWageSummary[ef10 == "männlich", LogHourlyWageSD], 
		#calculate proportion for employee share (female)     
	FemaleEmpolyeeShare = prop.table(mtable, 2)[, 2],  
		#put male wage mean value                            
	FemaleLogHourlyWageMean = lnWageSummary[ef10 == "weiblich", LogHourlyWageMean],
		#put male standard deviation    
	FemaleLogHourlyWageSD = lnWageSummary[ef10 == "weiblich", LogHourlyWageSD],
		#put TotalEmployee share here      
	TotalEmpolyeeShare = TotalEmpolyeeShare$Share,           
		#put Total mean wage here                     
	TotalLogHourlyWageMean = lnWageSummaryOverall$LogHourlyWageMean, 
		#put total standard deviation here                 
	TotalLogHourlyWageSD = lnWageSummaryOverall$LogHourlyWageSD,                   
	stringsAsFactors = FALSE)

#calculate Total line of before created variables 
Total = c("Total",     
		#sum over male shares (=1)   
	sum(lnWageSummaryTotal$MaleEmpolyeeShare),
		#build total of mean log wage male                              
	dat[!is.na(Group) & ef10 == "männlich", mean(lnWage, na.rm = T)], 
		#build total of standard deviation male      
	dat[!is.na(Group) & ef10 == "männlich", sd(lnWage, na.rm = T)],
		#sum over female shares (=1)      
	sum(lnWageSummaryTotal$FemaleEmpolyeeShare),                          
		#build total of mean log wage female
	dat[!is.na(Group) & ef10 == "weiblich", mean(lnWage, na.rm = T)],     
		#build total of standard deviation female
	dat[ef10 == "weiblich", sd(lnWage, na.rm = T)],
		#sum over all shares (=1 , no discromination in gender)	                         
	sum(lnWageSummaryTotal$TotalEmpolyeeShare),                             
		#build total of mean log wage of all observations
	dat[!is.na(Group), mean(lnWage, na.rm = T)],                            
		#build total of standard deviation of all observations
	dat[!is.na(Group), sd(lnWage, na.rm = T)])                              

#combine total table and summary table
lnWageSummaryTotal = rbind(lnWageSummaryTotal, Total)  
 
lnWageSummaryTotal[,2:10] = rapply(lnWageSummaryTotal[,2:10], as.numeric)
lnWageSummaryTotal = rapply(object = lnWageSummaryTotal, f = round, 
	classes = "numeric", how = "replace", digits = 2)	#round results

dat = data.frame(dat)   #put data back into data frame

#install and load xtable-package
install.packages("xtable")
library(xtable)

#print file with latex code
print(xtable(lnWageSummaryTotal, type = "latex"), 
	file = "covRegimeandLNWages.tex") 

\end{lstlisting}

\subsection{Code for Regression Analysis}
\subsubsection{Code for Quantlet 6}
\lstset{firstnumber = 415}
\begin{lstlisting}
#install and load dplyr- and stargazer-package
install.packages("dplyr")
library(dplyr)
install.packages("stargazer")
library(stargazer)

#OLS regression with 4 different specifications
model1 = lm (lnWage ~ SCTariffDummy + FCTariffDummy  + ef10 + east + ef9be + ef12be + ef26be + minimumWage + ef9 + educ2 + educ3 + shift + ef40 + agesq + ef41 + expsq + permanent , dat)

model2 = lm (lnWage ~ shareSC + shareFC + ef10 + east + ef9be + ef12be + ef26be + minimumWage + ef9 + educ2 + educ3 + shift + ef40 + agesq + ef41 + expsq + permanent , dat)

model3 = lm (lnWage ~ SCTariffDummy + shareSC + FCTariffDummy + shareFC + ef10 + east + ef9be + ef12be + ef26be + minimumWage + ef9 + educ2 + educ3 + shift + ef40 + agesq + ef41 + expsq + permanent , dat)

model4 = lm (lnWage ~ SCTariffDummy + shareSC + FCTariffDummy + shareFC + shareSCSC + shareFCFC + ef10 + east + ef9be + ef12be + ef26be + minimumWage + ef9 + educ2 + educ3 + shift + ef40 + agesq + ef41 + expsq + permanent , dat)

#output table result in latex code
stargazer(model1, model2, model3, model4, title="Results OLS Regression" ,
	keep = c("SCTariffDummy", "FCTariffDummy", "shareSC" , "shareFC" , "shareSCSC" , "shareFCFC" , "ef10") , 
	covariate.labels=c("Sectoral Contract","Firm Contract", "share SC","share FC","shareSCxSC","shareFCxFC" , "gender (male = 0)"),
	align=TRUE , omit.stat=c("ser","f"),  no.space=TRUE, out = "olsregression.tex")

### Quantile Regression ###

#install and load quantreg-package
install.packages("quantreg")
library(quantreg)

#free up additional memory
memory.limit(10000)   

#delete NAs from lnwage
quantileRegressionData   = dat %>% filter(!is.na(lnWage)) 

#set quantiles
quantile = seq(0.05, 0.95, by=0.05)   

#Quantile Regression with full data set
modelConditionalQR = rq(lnWage ~ SCTariffDummy + shareSC + FCTariffDummy + shareFC + shareFCFC + shareSCSC + ef10 + east+ ef9be + ef12be + ef26be + minimumWage + ef9 + educ2 + educ3 + shift + ef40 + agesq + ef41 + expsq + permanent , data=quantileRegressionData, tau = quantile)
quantreg.plot = (summary(modelConditionalQR))

#define a vector of which variables' coefficients should be plotted
plotvar = c(1, 2, 3, 4, 5, 6, 7, 8)  
plot(quantreg.plot, parm=plotvar)

modelConditionalQRCoef = modelConditionalQR[1]
modelConditionalQRCoef = as.data.frame(modelConditionalQRCoef)

#build vector with share for later calculation of the effects
calcAverage = c(lnWageSummaryTotal$TotalEmpolyeeShare[1],      
                lnWageSummaryTotal$TotalEmpolyeeShare[1],
                lnWageSummaryTotal$TotalEmpolyeeShare[2],
                lnWageSummaryTotal$TotalEmpolyeeShare[2])

#build data frame with results from conditional quantile regression
calcAverageCoefCQRSCSCFCFCQR = data.frame(
	tau10 = c(modelConditionalQRCoef[7, 2],  modelConditionalQRCoef[7, 2], modelConditionalQRCoef[6, 2],  modelConditionalQRCoef[6, 2]), 
	tau25 = c(modelConditionalQRCoef[7, 5],  modelConditionalQRCoef[7, 5], modelConditionalQRCoef[6, 5],  modelConditionalQRCoef[6, 5]), 
	tau50 = c(modelConditionalQRCoef[7, 10], modelConditionalQRCoef[7, 10], modelConditionalQRCoef[6, 10], modelConditionalQRCoef[6, 10]), 
	tau75 = c(modelConditionalQRCoef[7, 15], modelConditionalQRCoef[7, 15], modelConditionalQRCoef[6, 15], modelConditionalQRCoef[6, 15]), 
	tau90 = c(modelConditionalQRCoef[7, 18], modelConditionalQRCoef[7, 18], modelConditionalQRCoef[6, 18], modelConditionalQRCoef[6, 18]))

#calculate average partial effects
averagePartialEffectQR = data.frame(Quantiles = c("Sector Contract (SC)", "shareSC", "Firm Contract (FC)", "shareFC"), 
	tau10 = modelConditionalQRCoef[2:5, 2]  + (calcAverage * calcAverageCoefCQRSCSCFCFCQR$tau10),     
	tau25 = modelConditionalQRCoef[2:5, 5]  + (calcAverage * calcAverageCoefCQRSCSCFCFCQR$tau25), 
	tau50 = modelConditionalQRCoef[2:5, 10] + (calcAverage * calcAverageCoefCQRSCSCFCFCQ$tau50),
	tau75 = modelConditionalQRCoef[2:5, 15] + (calcAverage * calcAverageCoefCQRSCSCFCFCQ$tau75),
	tau90 = modelConditionalQRCoef[2:5, 18] + (calcAverage * calcAverageCoefCQRSCSCFCFCQ$tau90))                                                 

#print table in latex code
print(xtable(averagePartialEffectQR, type = "latex"), 
	file = "averagePartialEffectsCQR.tex")

### Uncondtional Quantile Regression ###

#install and load uuqr-package
install.packages("uqr")
library(uqr)

quantile2=c(0.1, 0.25, 0.5, 0.75, 0.9)
modelUnconditionalQR = urq(lnWage ~  SCTariffDummy + shareSC + FCTariffDummy + shareFC + shareFCFC + shareSCSC + ef10 + east + ef9be + ef12be + ef26be + minimumWage + ef9 + educ2 + educ3 + shift + ef40 + agesq + ef41 + expsq + permanent, data=quantileRegressionData, tau = quantile2 )

#calculate average partial effects for unconditional quantile regression:
modelUnconditionalQRCoef = modelUnconditionalQR[1]
modelUnconditionalQRCoef = as.data.frame(modelUnconditionalQRCoef)

#build data frame with results from unconditional quantile regression
calcAverageCoefUQRSCSCFCFC = data.frame(
	tau10 = c(modelUnconditionalQRCoef[7, 1], modelUnconditionalQRCoef[7, 1], modelUnconditionalQRCoef[6, 1], modelUnconditionalQRCoef[6, 1]),
	tau25 = c(modelUnconditionalQRCoef[7, 2], modelUnconditionalQRCoef[7, 2], modelUnconditionalQRCoef[6, 2], modelUnconditionalQRCoef[6, 2]), 
	tau50 = c(modelUnconditionalQRCoef[7, 3], modelUnconditionalQRCoef[7, 3], modelUnconditionalQRCoef[6, 3], modelUnconditionalQRCoef[6, 3]), 
	tau75 = c(modelUnconditionalQRCoef[7, 4], modelUnconditionalQRCoef[7, 4], modelUnconditionalQRCoef[6, 4], modelUnconditionalQRCoef[6, 4]), 
	tau90 = c(modelUnconditionalQRCoef[7, 5], modelUnconditionalQRCoef[7, 5], modelUnconditionalQRCoef[6, 5], modelUnconditionalQRCoef[6, 5]))

#calculate average partial effects
averagePartialEffectUQR = data.frame(Quantiles = c("Sector Contract (SC)", "shareSC", "Firm Contract (FC)", "shareFC"), 
	tau10 = modelUnconditionalQRCoef[2:5, 1] + (calcAverage * calcAverageCoefUQRSCSCFCFC$tau10), 
	tau25 = modelUnconditionalQRCoef[2:5, 2] + (calcAverage * calcAverageCoefUQRSCSCFCFC$tau25),
	tau50 = modelUnconditionalQRCoef[2:5, 3] + (calcAverage * calcAverageCoefUQRSCSCFCFC$tau50),
	tau75 = modelUnconditionalQRCoef[2:5, 4] + (calcAverage * calcAverageCoefUQRSCSCFCFC$tau75), 
	tau90 = modelUnconditionalQRCoef[2:5, 5] + (calcAverage * calcAverageCoefUQRSCSCFCFC$tau90))                                                 

#print table in latex code
print(xtable(averagePartialEffectUQR, type = "latex"), file = "averagePartialEffectUQR.tex") 

#calculate confidence intervalls  set for for bootstraping (bigger then 5)
modelUnconditionalQR.BCI = urqCI(modelUnconditionalQR , R=30 , seed=NULL , 
                                 colour=NULL , confidence=NULL , graph=TRUE , 
                                 cluster=NULL , BC=FALSE)


\end{lstlisting}